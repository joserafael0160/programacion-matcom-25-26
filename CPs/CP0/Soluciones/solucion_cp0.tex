\documentclass[12pt]{article}
\usepackage[T1]{fontenc}
\usepackage[utf8]{inputenc} % Eliminar si usas LuaLaTeX o XeLaTeX
\usepackage[spanish]{babel}
\usepackage{amsmath, amssymb}
\usepackage{enumitem}
\usepackage{tikz}
\usetikzlibrary{shapes, positioning, arrows.meta}
\usepackage{listings}
\usepackage{listingsutf8} % Para acentos en listings
\usepackage{xcolor}
\usepackage{geometry}
\usepackage[hidelinks]{hyperref}

% Geometry and lists
\geometry{a4paper, margin=1in}
\setlist[enumerate]{itemsep=2pt, topsep=5pt}
\setlist[itemize]{itemsep=2pt, topsep=5pt}

% Listings config
\lstset{
  inputencoding=utf8,
  language=Python,
  basicstyle=\ttfamily\small,
  keywordstyle=\color{blue},
  commentstyle=\color{green!60!black},
  stringstyle=\color{red},
  numbers=left,
  numberstyle=\tiny\color{gray},
  frame=single,
  breaklines=true,
  captionpos=b
}

\title{\textbf{Soluciones - CP0}}
\author{José Rafael Pérez Rivero C-122}
\date{29 octubre 2025}

\begin{document}

\maketitle

\section*{Problema 1: Las 12 Monedas}

\subsection*{Solución}

Dividimos las 12 monedas en tres grupos de 4: Grupo A, Grupo B, Grupo C.

\subsubsection*{Primera Pesada: A vs B}

\begin{itemize}
    \item \textbf{Caso 1: $A = B$} $\rightarrow$ La falsa está en C
    \item \textbf{Caso 2: $A < B$} $\rightarrow$ La falsa está en A (si es más ligera) o en B (si es más pesada)
    \item \textbf{Caso 3: $A > B$} $\rightarrow$ La falsa está en A (si es más pesada) o en B (si es más ligera)
\end{itemize}

\subsubsection*{Segunda Pesada (Caso $A = B$):}
Pesamos 3 monedas de C contra 3 monedas buenas de A:
\begin{itemize}
    \item Si se equilibran: La moneda restante en C es falsa (tercera pesada para determinar si es más pesada/ligera)
    \item Si $C < A$: Una de las 3 de C es más ligera
    \item Si $C > A$: Una de las 3 de C es más pesada
\end{itemize}

\subsubsection*{Tercera Pesada:}
Con solo 3 monedas sospechosas, pesamos 1 vs 1:
\begin{itemize}
    \item Si igualan: La tercera es falsa
    \item Si no: La más ligera/pesada según el caso anterior
\end{itemize}

\subsection*{Complejidad}
\begin{itemize}
    \item Número máximo de pesadas: 3
    \item Estrategia: \textbf{Búsqueda ternaria}
\end{itemize}

\section*{Problema 2: Torres de Hanoi}

\subsection*{Solución Matemática}

Definimos $T(n)$ como el número mínimo de movimientos para $n$ discos:

\[
T(n) = \begin{cases}
1 & \text{si } n = 1 \\
2T(n-1) + 1 & \text{si } n > 1
\end{cases}
\]

Resolviendo la recurrencia:
\[
T(n) = 2^n - 1
\]

\subsection*{Algoritmo Recursivo}

\begin{lstlisting}
def hanoi(n, origen, destino, auxiliar):
    if n == 1:
        print(f"Mover disco 1 de {origen} a {destino}")
        return
    # Mover n-1 discos de origen a auxiliar usando destino como auxiliar
    hanoi(n-1, origen, auxiliar, destino)
    # Mover el disco n de origen a destino
    print(f"Mover disco {n} de {origen} a {destino}")
    # Mover n-1 discos de auxiliar a destino usando origen como auxiliar
    hanoi(n-1, auxiliar, destino, origen)

# Ejemplo para 3 discos
hanoi(3, 'A', 'C', 'B')
\end{lstlisting}

\subsection*{Explicación del Algoritmo}
\begin{enumerate}
    \item \textbf{Caso base:} Si solo hay 1 disco, moverlo directamente del origen al destino
    \item \textbf{Paso recursivo:}
    \begin{itemize}
        \item Mover $n-1$ discos del origen al auxiliar (usando el destino como auxiliar temporal)
        \item Mover el disco $n$ (el más grande) del origen al destino
        \item Mover los $n-1$ discos del auxiliar al destino (usando el origen como auxiliar temporal)
    \end{itemize}
\end{enumerate}

\subsection*{Ejemplo para 3 discos}
\begin{verbatim}
Mover disco 1 de A a C
Mover disco 2 de A a B
Mover disco 1 de C a B
Mover disco 3 de A a C
Mover disco 1 de B a A
Mover disco 2 de B a C
Mover disco 1 de A a C
\end{verbatim}

\subsection*{Análisis}
\begin{itemize}
    \item \textbf{Tiempo:} $O(2^n)$
    \item \textbf{Espacio:} $O(n)$ (pila de recursión)
\end{itemize}

\section*{Problema 3: Los Puentes de Königsberg}
\subsection*{Modelado como Grafo}

\begin{center}
\begin{tikzpicture}[scale=0.8]
    \node[circle, draw, fill=blue!20] (A) at (0,2) {A};
    \node[circle, draw, fill=red!20] (B) at (2,2) {B};
    \node[circle, draw, fill=green!20] (C) at (2,0) {C};
    \node[circle, draw, fill=yellow!20] (D) at (0,0) {D};
    
    % Puentes
    \draw[thick] (A) to[bend left=10] node[midway,above] {1} (B);
    \draw[thick] (A) to[bend right=10] node[midway,below] {2} (B);
    \draw[thick] (A) to node[midway,left] {3} (D);
    \draw[thick] (B) to node[midway,right] {4} (D);
    \draw[thick] (B) to[bend left=10] node[midway,right] {5} (C);
    \draw[thick] (B) to[bend right=10] node[midway,left] {6} (C);
    \draw[thick] (C) to node[midway,below] {7} (D);
\end{tikzpicture}
\end{center}

\subsection*{Teorema de Euler}

Un grafo conexo tiene un \textbf{circuito euleriano} si y solo si:
\[
\forall v \in V, \quad \deg(v) \text{ es par}
\]

\subsection*{Aplicación al Problema}

Grados de los vértices:
\begin{align*}
\deg(A) &= 3 \quad \text{(impar)} \\
\deg(B) &= 5 \quad \text{(impar)} \\
\deg(C) &= 3 \quad \text{(impar)} \\
\deg(D) &= 3 \quad \text{(impar)}
\end{align*}

\textbf{Conclusión:} \textcolor{red}{No existe solución} - Demasiados vértices de grado impar.

\section*{Problema 4: Los Jarrones}

\subsection*{Solución Paso a Paso}

\begin{center}
\begin{tabular}{c|c|l}
Paso & Estado (5L, 3L) & Acción \\
\hline
0 & (0, 0) & Inicio \\
1 & (5, 0) & Llenar jarro de 5L \\
2 & (2, 3) & Verter 5L en 3L \\
3 & (2, 0) & Vaciar jarro de 3L \\
4 & (0, 2) & Verter 5L en 3L \\
5 & (5, 2) & Llenar jarro de 5L \\
6 & \textbf{(4, 3)} & Verter 5L en 3L (hasta llenar) \\
\end{tabular}
\end{center}

\subsection*{Generalización}

Para jarrones de capacidad $a$ y $b$, queremos medir $c$ litros. Solución existe si:
\[
\gcd(a, b) \mid c
\]

En nuestro caso: $\gcd(5, 3) = 1$ y $1 \mid 4$ $\rightarrow$ \textbf{Solución existe}.

\section*{Problema 5: Prisioneros y Sombreros}
\subsection*{Estrategia Óptima}

Los prisioneros acuerdan usar \textbf{código de paridad}:

\begin{itemize}
    \item \textbf{Prisionero 1} (el último): Cuenta sombreros \textbf{negros} que ve
    \begin{itemize}
        \item Si es \textbf{par}: Dice "BLANCO"
        \item Si es \textbf{impar}: Dice "NEGRO"
    \end{itemize}
    
    \item \textbf{Prisioneros $2, \ldots, n$}: 
    \begin{itemize}
        \item Escuchan todas las respuestas anteriores
        \item Cuentan sombreros negros que ven + información de paridad
        \item Deducen su propio sombrero con certeza
    \end{itemize}
\end{itemize}

\subsection*{Ejemplo con 4 Prisioneros}

Supongamos sombreros: [NEGRO, BLANCO, NEGRO, BLANCO]

\begin{enumerate}
    \item \textbf{P1} ve: [?, BLANCO, NEGRO, BLANCO] $\rightarrow$ 1 negro (impar) $\rightarrow$ Dice "NEGRO"
    \item \textbf{P2} oye "NEGRO", ve [?, NEGRO, BLANCO] $\rightarrow$ Deducen que tiene BLANCO
    \item \textbf{P3} oye "NEGRO, BLANCO", ve [?, BLANCO] $\rightarrow$ Deducen que tiene NEGRO  
    \item \textbf{P4} oye todo $\rightarrow$ Deducen que tiene BLANCO
\end{enumerate}

\subsection*{Análisis de Eficiencia}

\begin{itemize}
    \item \textbf{Prisionero 1}: 50\% de éxito
    \item \textbf{Prisioneros $2, \ldots, n$}: 100\% de éxito
    \item \textbf{Eficiencia total}: $n-1$ salvados garantizados
    \item \textbf{Técnica}: Codificación por paridad XOR
\end{itemize}

\end{document}